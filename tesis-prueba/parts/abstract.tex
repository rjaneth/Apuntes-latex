%\begin{otherlanguage}{spanish}
\begin{abstract}
\phantomsection
\addcontentsline{toc}{chapter}{Abstract}
%\addchaptertocentry{\abstrname}
%\addchaptertocentry{\abstractname} % Add the abstract to the table of contents
Edge detection has an essential role in post-processing of Polarimetric synthetic aperture radar (PolSAR) images. It is still a big challenge to extract all the edge features and suppress speckle noises, especially when weak/strong edges appear simultaneously inside and outside heterogeneous areas.

PolSAR images can provide more information than single-polarimetric synthetic aperture radar (SAR) images. As the prerequisite step of image processing, PolSAR edge detection is very important, which can provide important structural information for further object recognition and image interpretation of PolSAR images. However, a complex PolSAR scene usually includes both heterogenous and homogenous terrain types such as urban areas, forests, farmlands, waters, and so on.

In this thesis, we obtain the statistical properties (bias, variance) of edge point estimators in SAR/PolSAR images. In this way, we propose and evaluate new fusion and evidence selection techniques that take these properties into account.

\end{abstract}
%\end{otherlanguage}
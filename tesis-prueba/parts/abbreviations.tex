\thispagestyle{empty}
\chapter{Notaci�n}
\chaptermark{Notaci�n}
\newlist{abbrv}{itemize}{1}
\setlist[abbrv,1]{label=,labelwidth=2in,align=parleft,itemsep=0.05\baselineskip,leftmargin=!}
%\newlist{abbrv}{itemize}{1}
%\setlist[abbrv,1]{label=,labelwidth=1in,align=parleft,itemsep=0.1\baselineskip,leftmargin=!}
\textbf{Variables y funciones}
\begin{abbrv}
\item[$\alpha_i$]              Error Local de Discretizaci�n.
\item[$e_i$]                   Error global de discretizaci�n.
\item[$\Phi(t, y, h)$]         Funci�n incremento.
\item[$h$]                    Tama�o de paso.
%\item[$\Omega$]            	   Conjunto de regi�n de estabilidad absoluta.
\item[$\lambda$]               Valor propio.
\item[$G(t)$]                  Concentraci�n de Glucosa.
\item[$I(t)$]                  Concentraci�n de Insulina.
\item[$X(t)$]                  Insulina Activa.
\item[$G_b$]                   Glucosa basal.
\item[$I_b$]                   Insulina basal.
\end{abbrv}
\textbf{Operadores matem�ticos}
\begin{abbrv}
\item[$\left\|\cdot\right\|_2$]                  Norma euclidiana.
\item[$\left|\cdot\right|$]                      Valor absoluto.
\item[$\left\|\cdot\right\|_{\infty}$]           Norma infinito o norma m�ximo.
\end{abbrv}
\textbf{Abreviaturas}
\begin{abbrv}
\item[EDO]                  Ecuaci�n Diferencial Ordinaria.
\item[PTGO]                  Prueba de Tolerancia a la Glucosa Oral 
\item[PTGIV]                Prueba de Tolerancia a la Glucosa Intravenosa.
\item[CEH]                Clamp Eugluc�mico Hiperinsulin�mico.
\item[RK22]                  Runge-Kutta de orden dos.
\item[RK33]                  Runge-Kutta de orden tres.
\item[RK44]                 Runge-Kutta de orden cuatro.
\item[ABM4]                 Adams-Bashforth-Moulton de orden cuatro.
\item[$\mu$U/mL]            Micro-unidades por mililitro
\item[mg/dL]                 Miligramos por decilitro
 \end{abbrv}                 


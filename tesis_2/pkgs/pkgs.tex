%\usepackage[utf8]{inputenc}
%\usepackage[portugues,english]{babel}
%\usepackage[portuguese,english]{babel}
\usepackage{multicol}
\usepackage{xcolor}
\usepackage{subfigure}
%\usepackage[spanish]{babel}

\usepackage{graphicx}
%\usepackage{color}% para colocar la tabla
\usepackage{graphicx}
%\usepackage{float}
\usepackage{titlesec}
\usepackage[bookmarks,breaklinks,colorlinks=true,allcolors=black, citecolor=blue]{hyperref}
%\usepackage[hidelinks,colorlinks=true,linkcolor=blue,citecolor=blue]{hyperref}
\usepackage{listings}
\usepackage{inconsolata}
\usepackage{float}

%\usepackage[square,numbers]{natbib}
%\AtBeginDocument{
%  \renewcommand{\bibsection}{\chapter{\bibname}}
%} % Bibliografia en capitulo numerado

%\usepackage{geometry}
\usepackage{amsmath,amsthm}% cambia la fuente de la letra y ecuaciones con amsfonts,
%\usepackage{amsmath}  % For math
\usepackage{amssymb}  % For more math

\usepackage{parskip}
%\usepackage[official]{eurosym}
\usepackage{todonotes}
\usepackage{csquotes}

% new fuente
\usepackage{tgtermes}
\usepackage{float}
\usepackage{bm}
\usepackage{graphicx} % Required for including pictures
\usepackage{mathtools}
%\usepackage{multicol}
%\usepackage{caption} % For caption spacing
%\usepackage{subcaption} % For sub-figures

\usepackage{booktabs} % better tables
\usepackage{rotating} % landscape stuff, such as tables
\usepackage{array} % for m columns
\usepackage{multirow} % multi row tables
%\usepackage{bigdelim} % for big braces in tables (https://tex.stackexchange.com/a/129797/66561)

%\usepackage{afterpage} % for using \afterpage{\clearpage} before sidewaysfigures, to prevent them from going to the end:
%https://latex.org/forum/viewtopic.php?t=5903

%\usepackage[skip=-2pt, position=top, labelfont=normalfont]{subcaption} % multiple figures in one
%skip=-2pt reduces space between caption and subfigure
%singlelinecheck=false, justification=raggedright move the caption to the left

%\usepackage{microtype}%micro-typographic extension for better looks (subliminal refinements towards typographical perfection)
%%\usepackage{textcomp}% Solves some warnings....
%\usepackage[titletoc, title, header]{appendix}% For nicer appendices

%\usepackage{tabularx} % better tables

%\usepackage{multicol} % itemizations on multiple columns

%\newcommand{\HRule}{\rule{.9\linewidth}{.6pt}} % New command to make the lines in the title page
\newcommand{\decoRule}{\rule{.8\textwidth}{.4pt}} % New command for a rule to be used under figures
%\newcommand{\halfDecoRule}{\rule{.4\textwidth}{.4pt}} % New command for a rule to be used under figures

% con el comando \par empieza un nuevo parrafo con sangria, y con \\ comienza una nueva línea pero no un nuevo párrafo.
% las subsecciones empiezan sin sangria \parindent \indent
%\noindent to remove
% se establecerá en cero cuando se inserte un encabezado de sección
%\usepackage{indentfirst}
%set title spacing
%\usepackage{indentfirst}

\setlength\parindent{24pt}
\setlength{\parskip}{3.0pt}
\titlespacing*{\section}{0pt}{3.5ex plus 1ex minus .2ex}{2.3ex plus .2ex}
\titlespacing*{\subsection}{0pt}{3.5ex plus 1ex minus .2ex}{2.3ex plus .2ex}
%\usepackage{parskip}
%\setstretch{1.20}

% fancy chapter
\usepackage[Lenny]{fncychap}

%\usepackage{enumitem}%abreviaciones
%	HEADERS AND FOOTERS
%----------------------------------------------------------------------------------------
% abbreviations
%	ABBREVIATIONS PAGE DESIGN
%----------------------------------------------------------------------------------------
%--------------------------References
 %\usepackage[backend=biber, style=numeric, citestyle=nature]{biblatex}
 
% \usepackage{csquotes}
 % \usepackage[babel]{csquotes}
%\usepackage[style=numeric, maxbibnames=99, giveninits]{biblatex}

 % giveninits : es para colocar solo la inicial del nombre
 % giveninits : es para colocar solo la inicial del nombre

\usepackage[style=numeric-comp, sorting=none, backend=bibtex, doi=false, isbn=false, natbib=true, giveninits]{biblatex} % sorting=none ordena por orden de citas

\addbibresource{references.bib}

%\setlength\bibitemsep{5\itemsep}
\setlength{\bibitemsep}{\parskip}% para colocar espacio entre los items de referencias

% para colocar la tabla de revision bibliografica
%\usepackage{osameet3}
\usepackage{booktabs, longtable,array}
\usepackage{color, colortbl}
%\definecolor{name}{system}{definition}
\definecolor{Gray}{gray}{0.9}
%\usepackage[table]{xcolor}
\newcolumntype{P}[1]{>{\raggedright\arraybackslash}p{#1}}